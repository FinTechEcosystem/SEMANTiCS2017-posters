\documentclass[runningheads,a4paper]{llncs}
\usepackage{amssymb}
\setcounter{tocdepth}{3}
\usepackage{listings}
\usepackage{booktabs}
\usepackage{mathtools}
\usepackage{tabularx}
\usepackage{fixltx2e}
\PassOptionsToPackage{hyphens}{url}\usepackage{hyperref}
\usepackage[hyphens]{url}
\usepackage{upquote,textcomp}
\lstset{breaklines=true, basicstyle=\scriptsize\ttfamily, upquote=true}

\usepackage{fancyvrb}
\VerbatimFootnotes
\usepackage{cprotect}

\usepackage{graphicx}
\makeatletter
\def\maxwidth#1{\ifdim\Gin@nat@width>#1 #1\else\Gin@nat@width\fi}
\makeatother

\usepackage{amsmath}
\usepackage{pmml-new}

\usepackage{color,graphics,array,csscolor}

\usepackage{fontspec,unicode-math}
\usepackage[Latin,Greek]{ucharclasses}
\setTransitionsForGreek{\fontspec{Times New Roman}}{}

\usepackage{subscript}
\lstset{breaklines=true, basicstyle=\scriptsize\ttfamily}

\begin{document}
\mainmatter

\title{Adaptation of ontology sets for water related scenarios management with IoT systems for a more productive and sustainable agriculture systems}
\titlerunning{Adaptation of ontology sets for water r}
\author{Diego Sánchez de Rivera\inst{1} \and
Tomas Robles\inst{2} \and
Juan Antonio Lopez Morales\inst{3} \and
Azucena Sierra de Miguel\inst{4} \and
Mariano Navarro\inst{4} \and
María Sofía Iglesias Gómez\inst{4} \and
Juan Antonio Martinez Navarro\inst{5} \and
Antonio Skarmeta Gomez\inst{6}}
\authorrunning{Diego Sánchez de Rivera et al.}
\institute{Technical University of Madrid\and
Universidad Politécnica de Madrid\and
IMIDA\and
TRAGSA\and
Odin Solutions S.L.\and
Universidad de Murcia\\
\email{diegosanchez@dit.upm.es, 
trobles@dit.upm.es, 
juanantonio.lopez@carm.es, 
asdm@tragsa.es, 
mnc@tragsa.es, 
msig@tragsa.es, 
jamartinez@odins.es, 
skarmeta@um.es}}
\maketitle

\begin{abstract}
Water management is a key scenario for the deployment of IoT systems because of the particularities that arise depending on the geographical region as well as its inherent weather conditions. This scenario offers different and challenging problems to the deployment of IoT based applications and services which must rely on a rich technological vocabulary able to represent such characteristics and particularities. This is the reason why ontologies are the medium to achieve this goal. In this paper, we review the most well-known related ontologies as well as propose a model called MEGA promoted at state level with the objective of not only representing the information, but also integrating all the elements of an irrigation system, specially water distribution networks under interoperable platforms or systems

\keywords{IoT, Management, Ontology, agriculture}
\end{abstract}

\end{document}