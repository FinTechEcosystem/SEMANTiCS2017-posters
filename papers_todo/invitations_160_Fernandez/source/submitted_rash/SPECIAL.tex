\documentclass[runningheads,a4paper]{llncs}
\usepackage{amssymb}
\setcounter{tocdepth}{3}
\usepackage{listings}
\usepackage{booktabs}
\usepackage{mathtools}
\usepackage{tabularx}
\usepackage{fixltx2e}
\PassOptionsToPackage{hyphens}{url}\usepackage{hyperref}
\usepackage[hyphens]{url}
\usepackage{upquote,textcomp}
\lstset{breaklines=true, basicstyle=\scriptsize\ttfamily, upquote=true}

\usepackage{fancyvrb}
\VerbatimFootnotes
\usepackage{cprotect}

\usepackage{graphicx}
\makeatletter
\def\maxwidth#1{\ifdim\Gin@nat@width>#1 #1\else\Gin@nat@width\fi}
\makeatother

\usepackage{amsmath}
\usepackage{pmml-new}

\usepackage{color,graphics,array,csscolor}

\usepackage{fontspec,unicode-math}
\usepackage[Latin,Greek]{ucharclasses}
\setTransitionsForGreek{\fontspec{Times New Roman}}{}

\usepackage{subscript}
\lstset{breaklines=true, basicstyle=\scriptsize\ttfamily}

\begin{document}
\mainmatter

\title{SPECIAL: Scalable Policy-awarE Linked Data arChitecture for prIvacy, trAnsparency and
 compLiance}
\titlerunning{SPECIAL}
\author{Javier D. Fernández\inst{1} \and
Sabrina Kirrane\inst{2} \and
Axel Polleres\inst{1} \and
Rigo Wenning\inst{3}}
\authorrunning{Javier D. Fernández et al.}
\institute{Vienna University of Economics and Business & Complexity Science Hub, Vienna, Austria\and
Vienna University of Economics and Business, Vienna, Austria\and
W3C, Sophia-Antipolis, France\\
\email{jfernand@wu.ac.at, 
sabrina.kirrane@wu.ac.at, 
axel.polleres@wu.ac.at, 
rigo@w3.org}}
\maketitle

\begin{abstract}
SPECIAL is a research and innovation action, which is funded under the H2020
 ICT-18-2016 Big data PPP: privacy-preserving big data technologies call. The
 SPECIAL project aims to address the contradiction between Big Data innovation
 and privacy-aware data protection by proposing a technical solution that makes
 both of these goals realistic.

\keywords{Big Data, Compliance, Data protection, Legislation, Linked Data, Privacy, Robustness, Scalability, Transparency}
\end{abstract}


\section{Introduction}

The SPECIAL\footnote{https://www.specialprivacy.eu/} platform, which is routed in Semantic Web technologies and Linked Data principles  \cite{__RefHeading__354_1658379157}: (i) supports the acquisition
 of user consent at collection time and the recording of both data and metadata
 (consent, policies, event data, context) according to legislative and user-specified
 policies; (ii) caters for privacy-aware, secure workflows that include usage/access
 control, transparency and compliance verification; (iii) demonstrates robustness
 in terms of performance, scalability and security all of which are necessary to
 support privacy preserving innovation in Big Data environments; and (iv) provides a dashboard with feedback and control features that make privacy in Big
 Data comprehensible and manageable for data subjects, controllers, and processors. SPECIAL shall allow citizens and organisations to share more data, while
 guaranteeing data protection compliance, thus enabling both trust and the creation of valuable new insights from shared data. In order to support transparency
 across company borders Linked Data principles such as allocating unique IRIs
 to data resources and using these IRIs to associate metadata (i.e. policies and
 event data) with said resources will be employed.

\section{R\&D activities and goals}

The SPECIAL project, will be realised by combining and significantly extending big data architectures to handle Linked Data, harnessing them with sticky
 policies  \cite{__RefHeading__352_1658379157} as well as scalable queryable encryption  \cite{__RefHeading__356_1658379157} \cite{__RefHeading__350_1658379157}, and developing advanced user
 interaction and control features. SPECIAL builds on top of the Big Data Europe\footnote{https://www.big-data-europe.eu/} and PrimeLife\footnote{http://primelife.ercim.eu/} projects, exploits their results, and further advances the state of the art of privacy enhancing technologies. Against this background, the
 vision of SPECIAL translates into the following, concrete activities and goals:
\begin{itemize}
\item A policy management framework is required in order to ensure that data
 subjects can associate access and usage policies with their personal data and
 to support the derivation of policies for processed data. At the centre of
 this objective is the need for a policy language that is able to represent not
 only access/usage policies in machine readable format, but also legal rules,
 business rules, provenance data, and contextual information. Additionally,
 there is a need to develop automated policy synthesis techniques that can be
 used to derive policies for data produced by Big Data processing algorithms
 (aggregation, mining, etc.).
\item A transparency and compliance framework is needed in order to generate an
 immutable record of events that are linked to data and associated policies.
 Encryption, hashing and digital signatures are required in order to ensure
 both the integrity and non-repudiation of policies and events. In order to
 support traceability in terms of both the processing and sharing of personal
 data it is necessary to link data, policies and provenance/events with contextual information relating to the user and/or the environment. While, from a
 compliance perspective, it is necessary to automatically verify that the processing and sharing of personal data is inline with access and usage policies
 and also with the data protections legislation, and to inform the relevant
 parties in a nonintrusive manner.
\item Both the policy management and the transparency and compliance frameworks need to be realised in the form of a scalable architecture. The proposed
 scalable policy-aware Linked Data architecture will be evaluated under real-world conditions not only in terms of functionality, but also placing a strong
 emphasis on nonfunctional requirements, such as performance, scalability
 and security.
\end{itemize}

\subsubsection*{Acknowledgements.}Supported by the European Union's Horizon 2020 research and innovation programme under grant 731601.


\begin{thebibliography}{4}

\bibitem{__RefHeading__354_1658379157} Bizer, C., Heath, T., \& Berners-Lee, T. (2009). Linked Data - The Story So Far. International Journal
 on Semantic Web and Information Systems, 5, pp. 1-22.
\bibitem{__RefHeading__356_1658379157} Brakerski, Z., \& Vaikuntanathan, V. (2014). Efficient fully homomorphic encryption from (standard)
 LWE. SIAM Journal on Computing, 43(2), pp. 831-871.
\bibitem{__RefHeading__350_1658379157} Cash, D., Jaeger, J., Jarecki, S., Jutla, C., Krawczyk, H., Rosu, M., \& Steiner, M. (2014). Dynamic
 searchable encryption in very-large databases: Data structures and implementation. IACR Cryptology ePrint
 Archive, 2014:853
\bibitem{__RefHeading__352_1658379157} Trablesi, S., Neven, G., Ragget, D. et al. (2011). Report on design and implementation. PrimeLife
 Deliverable D5.3.4. Available at:
 \url{http://primelife.ercim.eu/images/stories/deliverables/d5.3.4-report\_on\_design\_and\_implementation-public.pdf}. 

\end{thebibliography}

\end{document}