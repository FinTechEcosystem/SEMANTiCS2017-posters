\documentclass[runningheads,a4paper]{llncs}
\usepackage{amssymb}
\setcounter{tocdepth}{3}
\usepackage{listings}
\usepackage{booktabs}
\usepackage{mathtools}
\usepackage{tabularx}
\usepackage{fixltx2e}
\PassOptionsToPackage{hyphens}{url}\usepackage{hyperref}
\usepackage[hyphens]{url}
\usepackage{upquote,textcomp}
\lstset{breaklines=true, basicstyle=\scriptsize\ttfamily, upquote=true}

\usepackage{fancyvrb}
\VerbatimFootnotes
\usepackage{cprotect}

\usepackage{graphicx}
\makeatletter
\def\maxwidth#1{\ifdim\Gin@nat@width>#1 #1\else\Gin@nat@width\fi}
\makeatother

\usepackage{amsmath}
\usepackage{pmml-new}

\usepackage{color,graphics,array,csscolor}

\usepackage{fontspec,unicode-math}
\usepackage[Latin,Greek]{ucharclasses}
\setTransitionsForGreek{\fontspec{Times New Roman}}{}

\usepackage{subscript}
\lstset{breaklines=true, basicstyle=\scriptsize\ttfamily}

\begin{document}
\mainmatter

\title{Building a High Quality Semantic Graph for Linked Science}
\titlerunning{Building a High Quality Semantic Graph }
\author{Ivelina Nikolova\inst{1}}

\institute{Ontotext\\
\email{ivelina.nikolova@ontotext.com}}
\maketitle

\begin{abstract}
Springer Nature SciGraph is a new Linked Open Data platform that aggregates and interlinks various data sources from the scholarly domain. It is part of Springer Nature's longstanding commitment to advance discovery by publishing robust and insightful research, supporting the development of new areas of knowledge, making ideas and information accessible around the world, and leading the way on open access. 

Some of the many benefits Springer Nature SciGraph offers are: 
\begin{itemize}
\item overcoming internal and external content silos in research communities; 
\item broadening users' perspective by semantic relations being revealed visually; 
\item encouraging developers to reuse Springer Nature's datasets; 
\item easily accessing high quality content from trusted and reliable sources; 
\item finding optimal content for analysis and recommendation tools for funders, librarians, conference organizers, etc.; 
\item increasing the discoverability of publications due to large parts of the datasets being freely accessible (CC BY-NC 4.0 license).
\end{itemize}

\keywords{LOD, linked open data, scigraph, springer nature}
\end{abstract}

\end{document}