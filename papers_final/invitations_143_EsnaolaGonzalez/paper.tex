\documentclass[runningheads,a4paper]{llncs}
\usepackage{amssymb}
\setcounter{tocdepth}{3}
\usepackage{listings}
\usepackage{booktabs}
\usepackage{mathtools}
\usepackage{tabularx}
\usepackage{fixltx2e}
\PassOptionsToPackage{hyphens}{url}\usepackage{hyperref}
\usepackage[hyphens]{url}
\usepackage{upquote,textcomp}
\lstset{breaklines=true, basicstyle=\scriptsize\ttfamily, upquote=true}

\usepackage{fancyvrb}
\VerbatimFootnotes
\usepackage{cprotect}

\usepackage{graphicx}
\makeatletter
\def\maxwidth#1{\ifdim\Gin@nat@width>#1 #1\else\Gin@nat@width\fi}
\makeatother

\usepackage{amsmath}
\usepackage{pmml-new}

\usepackage{color,graphics,array,csscolor}

\usepackage{fontspec,unicode-math}
\usepackage[Latin,Greek]{ucharclasses}
\setTransitionsForGreek{\fontspec{Times New Roman}}{}

\usepackage{subscript}
\lstset{breaklines=true, basicstyle=\scriptsize\ttfamily}

\begin{document}
\mainmatter

\title{Towards a Semantic Outlier Detection Framework in Wireless Sensor Networks}
\titlerunning{Towards a Semantic Outlier Detection Fr}
\author{Iker Esnaola-Gonzalez\inst{1} \and
Jesús Bermúdez\inst{2} \and
Izaskun Fernandez\inst{1} \and
Santiago Fernandez\inst{1} \and
Aitor Arnaiz\inst{1}}
\authorrunning{Iker Esnaola-Gonzalez et al.}
\institute{IK4-Tekniker\and
University of the Basque Country, UPV/EHU\\
\email{iker.esnaola@tekniker.es, 
jesus.bermudez@ehu.eus, 
izaskun.fernandez@tekniker.es, 
santiago.fernandez@tekniker.es, 
aitor.arnaiz@tekniker.es}}
\maketitle

\begin{abstract}
Outlier detection in the preprocessing phase of Knowledge Discovery in Databases (KDD) processes has been a widely researched topic for many years. However, identifying the potential outlier cause still remains an unsolved challenge even though it could be very helpful for determining what actions to take after detecting it. Furthermore, conventional outlier detection methods might still overlook outliers in certain complex contexts. In this article, Semantic Technologies are used to contribute overcoming these problems by proposing the SemOD (Semantic Outlier Detection) Framework. This framework guides the data-scientist towards the detection of certain types of outliers in WSNs (Wireless Sensor Network). Feasibility of the approach has been tested in outdoor temperature sensors and results show that the proposed approach is generic enough to apply it to different sensors, even improving the accuracy, specificity and sensitivity of outlier detection as well as spotting their potential cause.

\keywords{Knowledge Discovery in Databases, Outlier Detection, Semantic Technologies, Wireless Sensor Network}
\end{abstract}

\end{document}