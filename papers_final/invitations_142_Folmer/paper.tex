\documentclass[runningheads,a4paper]{llncs}
\usepackage{amssymb}
\setcounter{tocdepth}{3}
\usepackage{listings}
\usepackage{booktabs}
\usepackage{mathtools}
\usepackage{tabularx}
\usepackage{fixltx2e}
\PassOptionsToPackage{hyphens}{url}\usepackage{hyperref}
\usepackage[hyphens]{url}
\usepackage{upquote,textcomp}
\lstset{breaklines=true, basicstyle=\scriptsize\ttfamily, upquote=true}

\usepackage{fancyvrb}
\VerbatimFootnotes
\usepackage{cprotect}

\usepackage{graphicx}
\makeatletter
\def\maxwidth#1{\ifdim\Gin@nat@width>#1 #1\else\Gin@nat@width\fi}
\makeatother

\usepackage{amsmath}
\usepackage{pmml-new}

\usepackage{color,graphics,array,csscolor}

\usepackage{fontspec,unicode-math}
\usepackage[Latin,Greek]{ucharclasses}
\setTransitionsForGreek{\fontspec{Times New Roman}}{}

\usepackage{subscript}
\lstset{breaklines=true, basicstyle=\scriptsize\ttfamily}

\begin{document}
\mainmatter

\title{Linked Data at Kadaster}

\author{Erwin Folmer\inst{1} \and
Wouter Beek\inst{2}}

\institute{University of Twente\and
Dept. of Computer Science, VU University Amsterdam, NL\\
\email{e.j.a.folmer@utwente.nl, 
w.g.j.beek@vu.nl}}
\maketitle

\begin{abstract}
The poster will describe the architecture of one of the largest linked data implementations in the Netherlands, at the Kadaster. At Kadaster several key registers of the Netherlands,such as buildings and addresses, topographic maps, are published as linked data, and updated frequently. The demo will showcase interesting possibilities with this data, in so called data stories.

\keywords{Application, Architecture, Linked Data, Use Case}
\end{abstract}

\end{document}