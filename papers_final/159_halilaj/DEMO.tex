\documentclass[runningheads,a4paper]{llncs}
\usepackage{amssymb}
\setcounter{tocdepth}{3}
\usepackage{listings}
\usepackage{booktabs}
\usepackage{mathtools}
\usepackage{tabularx}
\usepackage{fixltx2e}
\PassOptionsToPackage{hyphens}{url}\usepackage{hyperref}
\usepackage[hyphens]{url}
\usepackage{upquote,textcomp}
\lstset{breaklines=true, basicstyle=\scriptsize\ttfamily, upquote=true}

\usepackage{fancyvrb}
\VerbatimFootnotes
\usepackage{cprotect}

\usepackage{graphicx}
\makeatletter
\def\maxwidth#1{\ifdim\Gin@nat@width>#1 #1\else\Gin@nat@width\fi}
\makeatother

\usepackage{amsmath}
\usepackage{pmml-new}

\usepackage{color,graphics,array,csscolor}

\usepackage{fontspec,unicode-math}
\usepackage[Latin,Greek]{ucharclasses}
\setTransitionsForGreek{\fontspec{Times New Roman}}{}

\usepackage{subscript}
\lstset{breaklines=true, basicstyle=\scriptsize\ttfamily}

\begin{document}
\mainmatter

\title{DemoEffTE: A Demonstrator of Dependency-aware Evaluation of Test Cases over Ontology}
\titlerunning{DemoEffTE}
\author{Lavdim Halilaj\inst{1} \and
Irlán Grangel-González\inst{1} \and
Maria-Esther Vidal\inst{2} \and
Steffen Lohmann\inst{3} \and
Sören Auer\inst{4}}
\authorrunning{Lavdim Halilaj et al.}
\institute{University of Bonn / Fraunhofer IAIS, Germany\and
Fraunhofer IAIS, Germany / Universidad Simon Bolivar, Venezuela\and
Fraunhofer IAIS, Germany\and
Technische Informationsbibliothek, Hannover, Germany\\
\email{halilaj@cs.uni-bonn.de, 
grangel@cs.uni-bonn.de, 
vidal@cs.uni-bonn.de, 
steffen.lohmann@iais.fraunhofer.de, 
soeren.auer@tib.eu}}
\maketitle

\begin{abstract}
Traditional approaches, which follow a {\em test-driven development} technique, allow a set of test cases to be exhaustively evaluated ensuring that each modification of an ontology does not violate predefined requirements. However, the time required for the evaluation of test cases is high and usually represents a bottleneck in an ontology development process. The {\em EffTE} framework tackles this problem; it relies on a graph-based model of the dependencies between test cases to support users during an ontology development process. Traversing the dependency graph is realized using breadth-first search along with a mechanism that tracks {\em tabu} test cases, i.e., test cases that will be ignored for further evaluation due to faulty parent test cases. As a result, the number of test cases that are evaluated is minimized, thus reducing the time required for validating an ontology after each modification. We demonstrate the benefits of prioritization and selection of the test cases to be evaluated with {\em DemoEffTE}. Attendees will observe the behavior of both a naive approach and the {\em EffTE }framework on different configuration settings such as different: (1) ontology size; (2) topology of the dependency graph of the test cases; and (3) number of test cases. The demo is available at: \url{http://vocol.iais.fraunhofer.de/DemoEffTE}.

\keywords{Dependency Graph, Ontology Engineering, Test Cases, Test-Driven Ontology Development}
\end{abstract}


\section{No heading specified}



\begin{abstract}
Traditional approaches, which follow a {\em test-driven development} technique, allow a set of test cases to be exhaustively evaluated ensuring that each modification of an ontology does not violate predefined requirements. However, the time required for the evaluation of test cases is high and usually represents a bottleneck in an ontology development process. The {\em EffTE} framework tackles this problem; it relies on a graph-based model of the dependencies between test cases to support users during an ontology development process. Traversing the dependency graph is realized using breadth-first search along with a mechanism that tracks {\em tabu} test cases, i.e., test cases that will be ignored for further evaluation due to faulty parent test cases. As a result, the number of test cases that are evaluated is minimized, thus reducing the time required for validating 