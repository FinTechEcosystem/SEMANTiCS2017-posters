\documentclass[runningheads,a4paper]{llncs}
\usepackage{amssymb}
\setcounter{tocdepth}{3}
\usepackage{listings}
\usepackage{booktabs}
\usepackage{mathtools}
\usepackage{tabularx}
\usepackage{fixltx2e}
\PassOptionsToPackage{hyphens}{url}\usepackage{hyperref}
\usepackage[hyphens]{url}
\usepackage{upquote,textcomp}
\lstset{breaklines=true, basicstyle=\scriptsize\ttfamily, upquote=true}

\usepackage{fancyvrb}
\VerbatimFootnotes
\usepackage{cprotect}

\usepackage{graphicx}
\makeatletter
\def\maxwidth#1{\ifdim\Gin@nat@width>#1 #1\else\Gin@nat@width\fi}
\makeatother

\usepackage{amsmath}
\usepackage{pmml-new}

\usepackage{color,graphics,array,csscolor}

\usepackage{fontspec,unicode-math}
\usepackage[Latin,Greek]{ucharclasses}
\setTransitionsForGreek{\fontspec{Times New Roman}}{}

\usepackage{subscript}
\lstset{breaklines=true, basicstyle=\scriptsize\ttfamily}

\begin{document}
\mainmatter

\title{Good Applications for Crummy Entity Linkers? The Case of Corpus Selection in Digital Humanities}
\titlerunning{Good Applications for Crummy Entity Lin}
\author{Alex Olieman\inst{1} \and
Kaspar Beelen\inst{1} \and
Jaap Kamps\inst{1} \and
Milan van Lange\inst{2}}
\authorrunning{Alex Olieman et al.}
\institute{University of Amsterdam\and
NIOD\\
\email{alex@olieman.net, 
kasparvonbeelen@gmail.com, 
kamps@uva.nl, 
m.van.lange@niod.knaw.nl}}
\maketitle

\begin{abstract}
Over the last decade we have made great progress in entity linking (EL) systems, but performance may vary depending on the context and, arguably, there are even principled limitations preventing a ``perfect'' EL system. This also suggests that there may be applications for which current ``imperfect'' EL is already very useful, and makes finding the ``right'' application as important as building the ``right'' EL system. We investigate the Digital Humanities use case, where scholars spend a considerable amount of time selecting relevant source texts. We developed WideNet; a semantically-enhanced search tool which leverages the strengths of (imperfect) EL without getting in the way of its expert users. We evaluate this tool in two historical case-studies aiming to collect a set of references to historical periods in parliamentary debates from the last two decades; the first targeted the Dutch Golden Age, and the second World War II. The case-studies conclude with a critical reflection on the utility of WideNet for this kind of research, after which we outline how such a real-world application can help to improve EL technology in general.

\keywords{Corpus Selection, Digital Humanities, Entity Linking, Interactive Information Retrieval, Real-World Applications, Semantically-Enhanced Search}
\end{abstract}

\end{document}