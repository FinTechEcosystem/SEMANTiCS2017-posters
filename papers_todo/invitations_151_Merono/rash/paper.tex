\documentclass[runningheads,a4paper]{llncs}
\usepackage{amssymb}
\setcounter{tocdepth}{3}
\usepackage{listings}
\usepackage{booktabs}
\usepackage{mathtools}
\usepackage{tabularx}
\usepackage{fixltx2e}
\PassOptionsToPackage{hyphens}{url}\usepackage{hyperref}
\usepackage[hyphens]{url}
\usepackage{upquote,textcomp}
\lstset{breaklines=true, basicstyle=\scriptsize\ttfamily, upquote=true}

\usepackage{fancyvrb}
\VerbatimFootnotes
\usepackage{cprotect}

\usepackage{graphicx}
\makeatletter
\def\maxwidth#1{\ifdim\Gin@nat@width>#1 #1\else\Gin@nat@width\fi}
\makeatother

\usepackage{amsmath}
\usepackage{pmml-new}

\usepackage{color,graphics,array,csscolor}

\usepackage{fontspec,unicode-math}
\usepackage[Latin,Greek]{ucharclasses}
\setTransitionsForGreek{\fontspec{Times New Roman}}{}

\usepackage{subscript}
\lstset{breaklines=true, basicstyle=\scriptsize\ttfamily}

\begin{document}
\mainmatter

\title{A Study of Intensional Concept Drift in Trending DBpedia Concepts}
\titlerunning{A Study of Intensional Concept Drift in}
\author{Albert Meroño-Peñuela\inst{1} \and
Efstratios Kontopoulos\inst{2} \and
Sándor Darányi\inst{3} \and
Yiannis Kompatsiaris\inst{4}}
\authorrunning{Albert Meroño-Peñuela et al.}
\institute{VU University Amsterdam\and
Information Technologies Institute, Centre for Research & Technology - Hellas\and
School of Library and Information Science, Göteborg University\and
CERTH - ITI\\
\email{albert.merono@vu.nl, 
skontopo@iti.gr, 
sandor.daranyi@hb.se, 
ikom@iti.gr}}
\maketitle

\begin{abstract}
Concept drift refers to the phenomenon that concepts change their intensional composition, and therefore meaning, over time. It is a manifestation of content dynamics, and an important problem with regard to access and scalability in the Web of Data. Such drifts go back to contextual influences due to social embedding as suggested by e.g. topic analysis, news detection, and trends in social networks. Using DBpedia as a source of timestamped Linked Open Data, we analyze the interaction between a sample of popular keywords, as recorded by Google Trends, and their respective concept drifts in DBpedia. For the latter task, we deploy SemaDrift, an ontology evolution platform for detecting and measuring content dislocation dependent on context modification. Our hypothesis is that social embedding and awareness is an important trigger for concept drift in crowdsourced knowledge bases on the Web.

\keywords{Concept Drift, DBpedia, Google Trends, Semantic Web, Wikipedia}
\end{abstract}

\end{document}